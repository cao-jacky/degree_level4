%%%%% Document Setup %%%%%%%%

\documentclass[12pt, onecolumn]{revtex4}    % Font size (12pt) and column number (one or two).

\usepackage[a4paper, left=2.5cm, right=2.5cm, top=2.5cm, bottom=2.5cm]{geometry}  % Defines paper size and margin length

\renewcommand{\baselinestretch}{1}     % Defines the line spacing

\usepackage{subcaption}

\usepackage[font=small, labelfont=bf]{caption}                      % Defines caption font size and caption title bolded
\captionsetup[figure]{justification=justified, singlelinecheck=off, font=footnotesize} 
\captionsetup{compatibility=false}

\usepackage{graphics,graphicx,epsfig,ulem}	% Makes sure all graphics works
\usepackage{amsmath} 						% Adds mathematical features for equations

\usepackage{etoolbox}                       % Customise date to preferred format
\makeatletter
\patchcmd{\frontmatter@RRAP@format}{(}{}{}{}
\patchcmd{\frontmatter@RRAP@format}{)}{}{}{}
\renewcommand\Dated@name{}
\makeatother

\usepackage[UKenglish]{babel}% http://ctan.org/pkg/babel

\usepackage{fancyhdr}

\pagestyle{fancy}                           % Insert header
\renewcommand{\headrulewidth}{0pt}
\lhead{\small Z0962251}  

\def\thesection{\arabic{section}}
\def\thesubsection{\alph{subsection}}

\def\bibsection{\section*{References}}        % Position reference section correctly
\setcitestyle{authoryear,round}
\setlength\bibhang{0.2in}
\usepackage[colorlinks]{hyperref}
\hypersetup{
    colorlinks=true,
    linkcolor=black,
    citecolor=black,    
    urlcolor=black,
}

%%%%% Document %%%%%
\begin{document}                     


\title{The measurement of the Hubble Constant: beyond the cosmic ladder} 
\date{Submitted: \today{}}
\author{Z0962251}

\maketitle
\thispagestyle{plain} % produces page number for front page

A precisely determined Hubble's constant $H_0$ would have an overarching effect on any feature of cosmological theory: the age of the Universe, the critical density of the Universe, or in the formation of cosmic structure. Producing a conclusive value for $H_0$ is difficult as absolute distances on the cosmic scale are difficult to measure. Inhomogeneous gravitational acceleration generates motion which does not follow the simple expansion as described by Hubble's Law $v=H_0 d$. An uncertainty arises due to the discrepancy between the methods to connect local distances to the smooth large-scale Hubble flow \citep{fukugita_cosmic}. \\

Several approaches for cosmic distance measurement should therefore be made to reduce systematic errors. These measurements can form the ``rungs'' of the \textit{cosmic distance ladder}, where large extragalactic distances ($>1000$ Mpc) are informed and calibrated by techniques which have smaller ranges. Astronomers may employ a variety of methods, therefore the ladder could be instead expressed as several pathways (Figure ). 

% insert diagram adapted from Jacoby paper

\newpage

\bibliographystyle{agsm}

\nocite{fukugita_cosmic}
\nocite{carroll_astro}
\nocite{jacoby_extragal}
\bibliography{cosmic_ladder}

\newpage

\end{document}