%%%%% Document Setup %%%%%%%%

\documentclass[12pt, onecolumn]{revtex4}    % Font size (12pt) and column number (one or two).

\usepackage[a4paper, left=2.5cm, right=2.5cm, top=2.5cm, bottom=2.5cm]{geometry}  % Defines paper size and margin length

\renewcommand{\baselinestretch}{1}     % Defines the line spacing

\usepackage{subcaption}

\usepackage[font=small, labelfont=bf]{caption}                      % Defines caption font size and caption title bolded
\captionsetup[figure]{justification=justified, singlelinecheck=off, font=footnotesize} 
\captionsetup{compatibility=false}

\usepackage{graphics,graphicx,epsfig,ulem}	% Makes sure all graphics works
\usepackage{amsmath} 						% Adds mathematical features for equations

\usepackage{etoolbox}                       % Customise date to preferred format
\makeatletter
\patchcmd{\frontmatter@RRAP@format}{(}{}{}{}
\patchcmd{\frontmatter@RRAP@format}{)}{}{}{}
\renewcommand\Dated@name{}
\makeatother

\usepackage[UKenglish]{babel}% http://ctan.org/pkg/babel

\usepackage{fancyhdr}

\pagestyle{fancy}                           % Insert header
\renewcommand{\headrulewidth}{0pt}
\lhead{\small Z0962251}  

\def\thesection{\arabic{section}}
\def\thesubsection{\alph{subsection}}

\def\bibsection{\section*{References}}        % Position reference section correctly
\setcitestyle{authoryear,round}
\setlength\bibhang{0.2in}
\usepackage[colorlinks]{hyperref}
\hypersetup{
    colorlinks=true,
    linkcolor=black,
    citecolor=black,    
    urlcolor=black,
}

%%%%% Document %%%%%
\begin{document}                     


\title{The measurement of the Hubble Constant: beyond the cosmic ladder} 
\date{Submitted: \today{}}
\author{Z0962251}

\maketitle
\thispagestyle{plain} % produces page number for front page

% Introduction to Hubble's constant and it's importance
A precisely determined Hubble's constant $H_0$ would have an overarching effect on any feature of cosmological theory from the age or critical density of the Universe to cosmic structure formation. Producing a conclusive value for $H_0$ is difficult as absolute distances on the cosmic scale are difficult to measure. Inhomogeneous gravitational acceleration generates motions which do not follow the simple expansion as described by Hubble's Law $v=H_0 d$. An uncertainty arises due to the discrepancy between the methods to connect local distances to the smooth large-scale Hubble flow \citep{fukugita_cosmic}. \\
% words: 86

% Cosmic distance ladder and cosmic paths
Several approaches for cosmic distance measurement should be utilised to reduce systematic errors. These measurements form the ``rungs'' of a \textit{cosmic distance ladder}, where large extragalactic distances ($>1000$ Mpc) are informed and calibrated by techniques with smaller ranges \citep{carroll_astro}. Astronomers often employ a variety of methods in tandem, therefore the ladder could be expressed as several pathways instead (Figure \ref{fig:cosmic_pathways}). 
% words: 60

% reiterate somewhere that galaxies should be distant so their peculiar velocities are small compared to the hubble flow

\begin{center}
\includegraphics[width=0.8\linewidth]{figures/cosmic_distance_pathways}
\captionof{figure}[Cosmic Distance Pathways]{Adapted from \cite{jacoby_extragal}, this diagram illustrates the various approaches to calculate $H_0$, each technique is roughly placed at the approximate range it operates at. One can see that there is not one strict ``cosmic ladder'', rather multiple pathways. For reference, the acronyms used are: B-W - Baade-Wessenlink; GCLF - Globular-Cluster Luminosity Function; LSC - Local Super Cluster; PNLF - Planetary Nebula Luminosity Function; SBF - Surface-Brightness Fluctuations; SG - Super Giant; SN - Supernovae; $\pi$ - parallax.}
\label{fig:cosmic_pathways}
\end{center}

% HST Key Project
The Hubble Space Telescope (HST) $H_0$ Key Project was an effort in the early 2000s to determine $H_0$ by calculating distances to Cepheid variables in local galaxies ($\le 20$ Mpc) then applying them as a calibration to 5 secondary independent distance indicators. Described by \cite{freedman_hstkeystone}, four of the methods (Type Ia supernovae, Tully-Fisher relation, surface-brightness fluctuations, and Type II supernovae) were able to produce $70\le H_0 \le72$ km s$^{-1}$ Mpc$^{-1}$ and the remaining technique (fundamental plane for elliptical galaxies) $H_0\approx82$ km s$^{-1}$ Mpc$^{-1}$. Over the next decade, the methodology would be refined and the sample of Cepheids and Type Ia supernovae improved (better observations) so $H_0=73.48 \pm1.66$ km s$^{-1}$ Mpc$^{-1}$ \citep{2011ApJ...730..119R, 2016ApJ...826...56R, 2018ApJ...855..136R}. These results set a standard benchmark for $H_0$, they were found by stepping along the cosmic ladder and whilst they are precise, it would be beneficial to directly calculate $H_0$ at large distances without the need for Cepheid-based calibrations. \\
% words: 142

% CMB anisotropies, Planck
An alternative is measuring Cosmic Microwave Background (CMB) anisotropies. Through analysing all-sky temperature and polarisation maps, $\Lambda$CDM cosmology models can be fitted which constrain cosmological parameters. Surveys of the CMB have included those performed by the spacecraft COBE, WMAP, and more recently, Planck. The results of the latter, $H_0=67.5\pm0.5$ km s$^{-1}$ Mpc$^{-1}$ \citep{2018arXiv180706209P}, is important as it is discrepant when compared to the HST-based result \citep{2018ApJ...855..136R}. Investigations of potential systematics in either methods have concluded that some arise due to the modelling of the Cepheids \citep{2018MNRAS.477.4534F} and there are residual systematics from certain spectra used in the Planck likelihood calculation \citep{2015PhRvD..91b3518S}. However the tension between the $H_0$ values still exists, therefore it would be beneficial to explore other methods which bypass the need for local distance calibration. \\
% words: 125

% Sunyaev–Zel'dovich effect, background
Considering the Sunyaev-Zel'dovich effect (SZE), this leads to a change in the apparent brightness of the CMB towards a cluster of galaxies or for any reservoir of hot plasma \citep{1999PhR...310...97B, 2002ARA&A..40..643C}. Combined with X-ray emission from intracluster gas, the SZE can be used as a tracer for cosmological parameters. \cite{1999PhR...310...97B} describes the technique as a comparison of the angular size of a galaxy cluster with the measure of the line-of-sight size of the cluster. With spectra data at hand, the emission of gas in a galaxy cluster can be described by the X-ray surface brightness, and the gas absorption by the measurement of the thermal SZE (an intensity change).  The surface brightness and intensity change can be re-expressed in terms of physical constants and angular structure factors, this then leads to a single expression for calculating the angular diameter distance, 
\begin{equation}
d_A = \Bigg( \frac{N^2_{SZ}}{N_X} \Bigg)\frac{\Lambda_{e0}}{4\pi (1+z)^3 [I_0 \Psi_0 \sigma_T]^2}.
\label{eqn:sze_ang}
\end{equation}
% words: 140

\textit{The full derivation and definitions for Equation \ref{eqn:sze_ang} can be found in \cite{1997ApJ...480..449H}.} \\

% Actual tests of the SZE and the SPT-SZ survey
Employing this equation with values for cluster redshifts $z$ and the deceleration parameter $q_0$, $H_0$ can be obtained in a direct and alternate way which is independent of the distance estimation chain. The South Pole Telescope Sunyaev-Zel'dovich (SPT-SZ) survey was a programme which made use of the SZE to detect galaxy clusters \citep{2009AIPC.1185..475C}. Multiple analyses have been performed to constrain cosmological parameters using the SPT-SZ datasets, the majority of which tested variations of the $\Lambda$CDM model \citep{2014ApJ...782...74H, 2016ApJ...832...95D}. \cite{2014ApJ...782...74H} examined models which were constraining single or double parameters, for example the neutrino mass. They combined external data sets (WMAP7, BAOs, previous $H_0$ calibrations) with their own and achieved $H_0=68.3\pm1.0$ km s$^{-1}$ Mpc$^{-1}$ which agrees with HST and Planck results.\\
% words: 115

% "advanced" level of technical detail wanted so perhaps discuss the intricacies/summarise how each method allows us to calculate H_0? 

% Introducing gravity as a tool and method for measuring H_0
The CMB can thus be utilised in multiple ways to calculate $H_0$. In a similar vein, the effects of the gravity can be explored and employed methods for measuring Hubble's constant without the need for the cosmic ladder. They manifest in the form of gravitational lensing and gravitational waves. \\
% words: 49

% Gravitational lensing 
Optical observations of the night sky sometimes reveal multiple arcs of light surrounding a central object, this effect is known as \textit{gravitational lensing}. It occurs when light travelling towards us from a distant bright object (a quasar) is curved by the space-time of a much massive object (a galaxy cluster or hypothetical MACHOs) in the foreground which lenses and arcs the light \citep{carroll_astro}. If the initial source, such as an active galactic nucleus or a supernova, varies in luminosity then this variability can be seen in the arcs, albeit with time delays as the light takes different paths \citep{suyu_2017}. This time delay can be related to the lens mass distribution and the ``time-delay distance'' $D_{\Delta t}$, where $D_{\Delta t}$ is the multiplicative combination of three angular diameter distances: observer-source distance $D_s$, observer-lens distance $D_d$, and lens-source distance $D_{ds}$ \citep{suyu_2017, 2018MNRAS.473..210S}. The application to cosmology arises as $D_{\Delta t}$ is inversely proportional to $H_0$ plus weakly dependent on other cosmological constants. \\
% words: 153

% How to make gravitational lensing a useful tool without the need for the cosmic ladder
\cite{2003ApJ...599...70K} presents mass models which were developed to reduce known systematics such as the radial mass profile, dust extinction, etc. Three particular mass models (SIE, SPLE1, SPLE2) were tested which considered only the gravitational lensing constraints and so examined no stellar dynamics. This resulted in $H_0$ ranging from $71$--$74$ km s$^{-1}$ Mpc$^{-1}$ and a best value of $H_0=74^{+10}_{-11}$ km s$^{-1}$ Mpc$^{-1}$. It is clear that whilst the uncertainties are large, the general $H_0$ value agrees with the Cepheid/Planck results. Introducing additional constraints would improve the precision, \cite{2011A&A...536A..53C} demonstrates that parameters such as the baryonic fraction in the Einstein radius and the velocity dispersion of the lensing galaxy could be found by combining spatially deconvolved HST F160W images with VLT spectroscopic data. \\
% words: 118

% Gravitational lensing is okay, but how about gravitational waves instead
Gravitational lensing is therefore a highly viable method in the measurement of $H_0$. However it has the limitation of requiring long-periods of photometric observations to observe the time delays. A more suitable strategy would be to utilise gravitational waves as a standard siren as observatories have been purposely built for their detection. Waves originating from the decaying orbit of an ultra-compact, binary neutron star system would be the most likely to be registered by Earth based detectors \citep{Schutz:1986aa}. \\
% words: 78

% Gravitational waves
\cite{2017Natur.551...85A} describes their approach used to calculate $H_0$ for object GW170817 detected by the Advanced Laser Interferometer Gravitational-wave Observatory (LIGO) \citep{2015CQGra..32g4001L} and the Virgo detector \citep{2015CQGra..32b4001A}. The gravitational wave (GW) data is used to infer the distance $d$ to the source through constraining a posterior probability in a Bayesian framework model. An initial posterior distribution for the observed data $x_{GW}$ can be converted into a posterior on the inclination angle $\cos{i}$ and $H_0=v_H/d$, where $v_H$ is the Hubble flow velocity. $\cos{i}$ is important as it was found that $d$ is strongly correlated with the inclination of the binary orbital plane. Obtaining $v_H$ for the source, the host galaxy's measured recessional velocities can be corrected for local peculiar motions. Together, this method does not require the Hubble flow velocities of any local calibrating galaxies which have been estimated using the distance ladder. \\
% words: 133

% Estimate of $H_0$ from the gravitational waves method performed 
Applying $v_H=3017\pm166$ km s$^{-1}$ to the model produces a maximum a posteriori value of $H_0=70.0^{+12.0}_{-8.0}$ km s$^{-1}$ Mpc$^{-1}$ which is consistent with the known results. The precision of the GW method could be improved by employing more detectors in the observation network. KAGRA is a Japanese GW telescope which is currently under construction, it features cryogenic cooling and it is based underground \citep{1742-6596-610-1-012016}, both of these will reduce the thermal and seismic noise therefore improving the data signal-to-noise ratio. Alternatively, the precision could be refined simply by incorporating alternative data-sets. Recent work by \cite{2018arXiv180610596H} demonstrate that the uncertainty in the \cite{2017Natur.551...85A} $H_0$ value is dominated by the degeneracy in the GW signal between the source distance and the ``weakly constrained'' viewing angle. They provide an alternative analysis which makes use of a collection of radio images for GW170817's superluminal jets to constrain the inclination angle. Employing analytical modelling, full hydrodynamic numerical simulations and semi-analytic calculations of synthetic jet models, the new constraints resulted in an improved measurement of  $H_0=68.9^{+4.7}_{-4.6}$ km s$^{-1}$ Mpc$^{-1}$. \\
% words: 168

% Summary of the different methods, from cosmic ladder to gravitation waves
% Table including all the various methods found and the associated Hubble constant values
\begin{center}
\renewcommand{\arraystretch}{1.0}
\begin{tabular}{c@{\hskip 20pt}c} 
 \hline
 \textbf{Method} & $\boldsymbol{H_0}$ \textbf{(km s$^{-1}$ Mpc$^{-1}$)} \\ [0.5ex] 
 HST Key Project with Cepheids+secondary & $73.48\pm1.66$ \\
 Planck observations of the CMB & $67.5\pm0.5$\\
 Sunyaev-Zel'dovich Effect and SPT-SZ data & $68.3\pm1.0$ \\
 Gravitational lensing and mass modelling & $74^{+10}_{-11}$ \\
 Gravitational waves with LIGO and Virgo & $70.0^{+12.0}_{-8.0}$ \\
 Gravitational waves plus radio data for jets & $68.9^{+4.7}_{-4.6}$ \\
 \hline
\end{tabular}
\captionof{table}[Hubble's Constant Summary Table]{Summarised results from the various methods used to calculate Hubble's Constant. All values are in general agreement, however there is tension between the results found from Cepheid calibrated values and the CMB Planck results. The alternative methods appear to support the Planck value for $H_0$.}
\label{table:spectral_classification}
\end{center}

% Conclusion
In conclusion, this discussion has detailed three alternative methods which can be used to measure Hubble's constant $H_0$ without the need to climb the cosmic ladder in the local Universe. Exploiting and measuring the Sunyaev-Zel'dovich effect, gravitational lensing, and gravitational waves leads to accurate values for $H_0$ which agree with those found with the traditional standard candle techniques. If used in tandem, one could potentially produce an extremely accurate and precise value. Measuring Hubble's constant beyond the cosmic ladder is a promising field of research and with more observations and analysis there is no doubt that eventually a single unifying value for $H_0$ will be found. \\ 
% words: 103

\textit{Final word count: 1470}

\newpage

\bibliographystyle{agsm}

%\nocite{fukugita_cosmic}
%\nocite{carroll_astro}
%\nocite{jacoby_extragal}
\bibliography{cosmic_ladder}

\newpage

\end{document}