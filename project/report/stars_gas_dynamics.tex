%%%%% Document Setup %%%%%%%%

\documentclass[12pt, onecolumn]{revtex4}    % Font size (12pt) and column number (one or two).

\usepackage[a4paper, left=2.5cm, right=2.5cm,
 top=2.5cm, bottom=2.5cm]{geometry}       % Defines paper size and margin length

\renewcommand{\baselinestretch}{1}     % Defines the line spacing

\usepackage[font=small,
labelfont=bf]{caption}                      % Defines caption font size and caption title bolded

\usepackage{graphics,graphicx,epsfig,ulem}	% Makes sure all graphics works
\usepackage{amsmath} 						% Adds mathematical features for equations

\usepackage{etoolbox}                       % Customise date to preferred format
\makeatletter
\patchcmd{\frontmatter@RRAP@format}{(}{}{}{}
\patchcmd{\frontmatter@RRAP@format}{)}{}{}{}
\renewcommand\Dated@name{}
\makeatother

\usepackage{fancyhdr}


\pagestyle{fancy}                           % Insert header
\renewcommand{\headrulewidth}{0pt}
\lhead{\small Jacky Cao}                        
\rhead{\small The relation between stars and gas in distant galaxies}                

\def\thesection{\arabic{section}}

\def\bibsection{\section*{References}}        % Position reference section correctly


%%%%% Document %%%%%
\begin{document}                     


\title{The relation between stars and gas in distant galaxies} 
\date{Submitted: \today{}}
\author{Jacky Cao}
\affiliation{\normalfont Level 4 Project, MPhys Theoretical Physics\\ Supervisor: Dr. Mark Swinbank\\ Department of Physics, Durham University}

\begin{abstract}              
 
 Observing any galaxy in the universe will yield the fact that it contains stars and also gas. The dynamics of both can be explored by observing galaxies and collecting spectroscopic data. 
 
Abstract abstract abstract abstract abstract abstract abstract abstract abstract abstract abstract abstract abstract abstract abstract abstract abstract abstract abstract abstract abstract abstract abstract abstract abstract abstract abstract abstract abstract abstract abstract abstract abstract abstract abstract abstract abstract abstract abstract abstract abstract abstract abstract abstract abstract abstract abstract abstract abstract abstract abstract abstract abstract abstract 

\end{abstract}


\maketitle
%\thispagestyle{plain} % produces page number for front page

\tableofcontents
\let\toc@pre\relax
\let\toc@post\relax

\newpage

\section{Introduction} 

Amongst the different types of cosmic structure within our universe, galaxies can be seen as the island powerhouses of industry and activity. Containing countless stars, gas, dust, and dark matter \cite{carroll_astro}, it would be difficult not to express the statement that the internal motions of these objects must be linked in some form of a galactic relationship. 

By utilising astronomy's most powerful tool, observation, galaxies, their structure and the motions of the objects within them can be studied to a great depth. Say if we took optical measurements of the star population, then we can understand the potential age of the galaxy. [ref] Or if we wanted to know about the material composition or the distance to that galaxy, we could split the light which we receive in a spectrograph.

Gathering

\subsection{Galaxy classification}

Galaxies themselves can be grouped and categorised together in the \textit{Hubble Sequence} or the \textit{Hubble Tuning Fork}. With a horizontal handle and two prongs, the sequence itself does not show the evolution of the galaxies, rather it provides a way to view the possible different types of galaxies on one graph. [REF] 

Hubble sequence.

What do I want to link into? Galaxies. Something to do with how they house stars and can lead to life. the importance of understanding their dynamics

What do I want to say with this? I want to introduce galaxies, the different types of galaxies, how they form, how they can be confused with other types of structure. 

\section{Data} 

Duis eget tellus tortor. Cum sociis natoque penatibus et magnis dis parturient montes, nascetur ridiculus mus. In tellus nulla, sodales eu pulvinar at, accumsan quis magna. Nunc sed lacus diam. Nam enim mauris, imperdiet ut egestas quis, tincidunt at odio. Ut viverra nulla at libero dictum aliquet. Suspendisse lacus lacus, imperdiet nec elit nec, ullamcorper facilisis ex. 

\subsection{Subsection heading} 

Proin sit amet mauris tincidunt, consectetur nisi ultrices, dapibus elit. Nullam vitae faucibus odio, pharetra ultrices tortor. Class aptent taciti sociosqu ad litora torquent per conubia nostra, per inceptos himenaeos. 

\section{Analysis} 

Duis eget tellus tortor. Cum sociis natoque penatibus et magnis dis parturient montes, nascetur ridiculus mus. In tellus nulla, sodales eu pulvinar at, accumsan quis magna. Nunc sed lacus diam. Nam enim mauris, imperdiet ut egestas quis, tincidunt at odio. Ut viverra nulla at libero dictum aliquet. Suspendisse lacus lacus, imperdiet nec elit nec, ullamcorper facilisis ex. 

\section{Discussion} 

Duis eget tellus tortor. Cum sociis natoque penatibus et magnis dis parturient montes, nascetur ridiculus mus. In tellus nulla, sodales eu pulvinar at, accumsan quis magna. Nunc sed lacus diam. Nam enim mauris, imperdiet ut egestas quis, tincidunt at odio. Ut viverra nulla at libero dictum aliquet. Suspendisse lacus lacus, imperdiet nec elit nec, ullamcorper facilisis ex. 

\section{Conclusions}
 
Donec finibus, tellus sit amet luctus sodales, lectus ante accumsan ligula, at condimentum lorem justo a sapien. Phasellus vel tortor vitae metus lacinia efficitur ac vel ex. Aenean eget congue leo. Aliquam cursus mauris sit amet arcu dignissim, vel condimentum nisi sodales. 

\begin{acknowledgments}
(OPTIONAL) The author would like to thank...
\end{acknowledgments}

\bibliographystyle{unsrt}
\bibliography{stars_gas_dynamics}

\end{document}