%%%%% Document Setup %%%%%%%%

\documentclass[12pt, onecolumn]{revtex4}    % Font size (12pt) and column number (one or two).

\usepackage[a4paper, left=2.5cm, right=2.5cm, top=2.5cm, bottom=2.5cm]{geometry}  % Defines paper size and margin length

\renewcommand{\baselinestretch}{1}     % Defines the line spacing

\usepackage{subcaption}

\usepackage[font=small, labelfont=bf]{caption}                      % Defines caption font size and caption title bolded
\captionsetup[figure]{justification=justified, singlelinecheck=off, font=footnotesize} 
\captionsetup{compatibility=false}

\usepackage{graphics,graphicx,epsfig,ulem}	% Makes sure all graphics works
\usepackage{amsmath} 						% Adds mathematical features for equations

\usepackage{etoolbox}                       % Customise date to preferred format
\makeatletter
\patchcmd{\frontmatter@RRAP@format}{(}{}{}{}
\patchcmd{\frontmatter@RRAP@format}{)}{}{}{}
\renewcommand\Dated@name{}
\makeatother

\usepackage{fancyhdr}

\pagestyle{fancy}                           % Insert header
\renewcommand{\headrulewidth}{0pt}
\lhead{\small }                        
\rhead{\small }                

\def\thesection{\arabic{section}}
\def\thesubsection{\alph{subsection}}

\def\bibsection{\section*{References}}        % Position reference section correctly
\setcitestyle{authoryear,round}
\setlength\bibhang{0.2in}
\usepackage[colorlinks]{hyperref}
\hypersetup{
    colorlinks=true,
    linkcolor=black,
    citecolor=black,    
    urlcolor=black,
}

\usepackage{tabularx}

%%%%% Document %%%%%
\begin{document}                     

\title{Prediction limitations with the spring predictability barrier in the Zebiak-Cane model for El Ni\~{n}o Southern Oscillations} 
%\date{Submitted: \today{}} \author{Jacky Cao}

\maketitle
\thispagestyle{plain} % produces page number for front 

\section{Introduction}
% Introducing ENSOs and what EN and SO are individually and then setting the problem from the offset 
The El Ni\~{n}o Southern Oscillations (ENSOs) are generally known as a composite weather phenomena originating in the Pacific Ocean producing lasting teleconnections on the global climate system. The El Ni\~{n}o component of ENSO can be approximately considered to be an oceanic warming event which disrupts the normal Pacific circulation at irregular intervals of 2--7 years, whilst the Southern Oscillations are an inter-annual flip of the tropical sea level pressure between the western and eastern Pacific leading to the weakening and strengthening of the easterly trade winds across the ocean. To produce a conclusive theory for ENSOs one must be able to describe and understand the complete underlying mechanisms. One such hypothesis has yet to arise, however various attempts have been made to comprehend individual components and effects. \\
% words: 127

% How Bjerknes, Zebiak and Cane set the stage for ENSO research 
\cite{doi:10.1175/1520-04931969097} first theorised that a positive ocean-atmosphere feedback system would result in an El Ni\~{n}o event. An initial positive sea surface temperature (SST) anomaly in the eastern Pacific would reduce the east-west SST gradient which leads to the strengthening of the Walker circulation and thus the production of weaker trade winds across the equatorial Pacific. In a complete ENSO theory this positive system would be counterbalanced by a negative loop which returns the Pacific to its ``normal'' (pre-ENSO) state. Whilst Bjerknes' hypothesis failed to provide a negative feedback mechanism, \cite{Zebiak:1987aa} presented a model which demonstrated and outlined the coupling between the atmosphere and the ocean to produce an ENSO event. The atmospheric component used was a linear Gill-type model \citep{Gill:1980aa} which describes the atmosphere's response to SST anomalies, and the ocean represented by a low-gravity system which is forced by the wind stress from the atmospheric constituent. \\
% words: 147

With their model they were able to replicate features observed during ENSO events such as equatorial westerly wind anomalies in the central Pacific and large SST anomalies in the eastern Pacific, on top of that they were able to predict the onset of the 1986--1987 and 1991--1992 ENSO events. Despite this success, they recognised their limited ability in simulating the real complete system as detailed comparisons with observational data would reveal discrepancies in their atmospheric and oceanic simulations. Furthermore, the short warm episodes in 1993 and 1994 would be missed in the predictions made with the Zebiak-Cane (ZC) model. This therefore requires more sophisticated models to better describe and forecast ENSO events.
% words: 112

\section{Prediction limitations and the Zebiak-Cane Model}

The prediction of ENSO events is particularly difficult as there are generally two types of El Ni\~{n}o events to account for. The first are canonical events which generally develop along the South American coast and then propagate westwards across the Pacific, ``Eastern-Pacific'' events \citep{rasmusson1982variations}. The second type of events have non-propagating warm SST concentrated mostly in the central Pacific, ``Central-Pacific'' events \citep{ashok2007nino}. In an attempt to test whether the ZC model can predict either types of events, it was demonstrated by \cite{duan2013behaviors} that the model tended to do well whilst simulating Eastern-Pacific (EP) events and functioned badly when reproducing Central-Pacific (CP) events. This indicates that the ZC model may just contain the physics to explain EP events and that alterations would be required to additionally account for other events. \\
% words: 129

Additional challenges for ENSO forecasting arise due to the nonlinear and complex coupling between the ocean and atmospheric systems. Within the ZC model this relationship was constrained as a result of the researchers' initial assumptions and parameter choices when constructing their theory. For example with the simulation of monthly mean SST anomalies in the atmosphere model, concessions had to be made to ensure accurate results could be produced whilst the analysis not being computationally costly. Further experimentation would exhibit that the amplitude and time scale of the ENSO cycles would be sensitive to changes within the coupled mathematical model: an increase (or decrease) in the strength of the coupling between the atmosphere and ocean would lead to increase (or decrease) in the amplitudes and periods. \\
% words: 125

It is therefore evident that the ZC model is a simplification of the real climate system and improvements must be made to provide a better fit with the observational data. Theoretical modifications of the individual atmospheric and oceanic components plus of their inter-relationship would be key in developing the theory further. Several modern ENSO oscillator theories (Table \ref{table:enso_oscillators}) employ the ZC model as a basic foundation whilst making adjustments to parts of the mathematical modelling thus improving their ability to explain and account for various ENSO effects.
% words: 87

\begin{table}[htbp]
\renewcommand{\arraystretch}{1.0}
%\begin{tabular}{c@{\hskip 20pt}c} 
\begin{tabular}{|p{7cm}|p{9cm}|}
 \hline
 \textbf{Theory} & \textbf{Main components} \\ [0.5ex] 
 \hline
 The Delayed Oscillator \citep{Suarez:1988aa, Battisti:1988aa} & Considers the effects of equatorially trapped oceanic wave propagation. \\
 \hline
 The Recharge Oscillator \citep{Jin:1997aa} & Considers the buildup of warm water in the western Pacific as a precondition to the development of El Ni\~{n}o. \\
 \hline
 The Western Pacific Oscillator \citep{Weisberg:1997aa, wang1999effects} & Considers the role of the western Pacific and off-equatorial SST SST anomalies in the western Pacific. \\
 \hline
 The Advective-Reflective Oscillator \citep{Picaut663} & Considers the importance of the positive feedback of zonal currents that advect the western Pacific warm pool towards the east during El Ni\~{n}o.  \\
 \hline
 The Unified Oscillator \citep{wang2001unified} & Considers dynamics and thermodynamics of a coupled ocean-atmosphere system which is similar to Zebiak-Cane. \\
 \hline
\end{tabular}
\captionof{table}[ENSO Oscillators]{Various ENSO oscillator theories and their main differences from the ZC model.}
\label{table:enso_oscillators}
\end{table}

Changes to the ZC model are not limited to just the atmospheric and oceanic components, as a response to the model failing to agree with observations from 1992 to 1995, \cite{qian1997multiple} introduced planetary scale Hadley and Walker cells which improved the prediction of equatorial eastern Pacific SST anomalies for 1970--1971 and in 1992--1995. \\
% words: 53

However further shortfalls exist within the modelling for ENSOs which limits their accuracy and applicability. In particular, there is a limitation found in many models known as the ``spring predictability barrier'' (SPB) where the seasonal predictions for ENSOs made during or before boreal spring (March--May) have much lower skill than those made at other times of the year \citep{torrence1998annual}. This barrier can be evidenced in oceanic circulation models \citep{latif1992much}, in dynamical-statistical models \citep{balmaseda1994enso}, and in coupled ocean-atmosphere models \citep{goswami1991predictability, xue1994prediction}. To better predict ENSO events, this barrier therefore must be understood and accounted for within climate modelling.
% words: 97

\section{Evidence for the spring predictability barrier}

Examinations of the correlation between ENSO data sets through autocorrelation and persistence provides one method for identifying the predictability barrier \citep{torrence1998annual}. \\
% words: 21

The autocorrelation of a time series is the correlation between itself and a copy of the data which has been time-lagged. Analyses of monthly sea level pressures across the Pacific Ocean yields the SST and Southern Oscillation Index (SOI) to have high autocorrelations which agrees with the general observation that ENSO events normally persist for several months \citep{trenberth1976spatial}. \\
% words: 58

Persistence focuses on the fixed-phase correlation between different months within a single time series. Contrasting with autocorrelation which is independent of the starting month, persistence shows any seasonal changes in the correlations between one month and the next \citep{troup1965southern}. Analysis by \cite{torrence1998annual} of NINO3 SST and GMSLP SOI data shows that persistence has distinct, regular structure which is phase locked to an annual cycle (Fig. \ref{fig:persistence_sst_soi}). Regardless of starting month, the persistence has a rapid decline in the March--April--May period which can be attributed as the manifestation of the spring predictability barrier. \\
% words: 92

\begin{figure}
\includegraphics[width=\textwidth]{data/persistence_sst_soi}
\caption[Persistence]{Data showing persistence between different months of the year \citep{torrence1998annual}: (a) Persistence of NINO3 surface sea temperature (SST) with each curve shifted to line up with the starting month on the top axis (JAJO=January, April, July, October), and the corresponding lag month on the lower axis. The black dots show the lag-1 persistence, and all twelve curves for one year were repeated for clarity; (b) Same analysis applied but for the GMSLP Southern Oscillation Index (SOI). }
\label{fig:persistence_sst_soi}
\end{figure}

With the SPB apparent from observational data, it would be logical to consider where the barrier occurs in theoretical modelling. Providing an initial look at dynamical and statistical models, these frameworks often have origins in seasonal forecasting making them ideal in ENSO forecasting as they couple the dynamics of the atmosphere, ocean and land. \\
% words: 54

\cite{jan2005did} tests the skill of the six models adapted from research by the European Centre for Medium-Range Weather Forecasts (ECMWF). What became apparent was whilst the SPB does arise in the modelling, it appears that they emerge at different times within the year (Fig. \ref{fig:ecmwf_plot}). However this may just be an effect of testing the ECMWF models individually, if combinations of the models (multi-model ensembles, MMEs) were to be applied, the results may better align with the data and the variability in barrier period may reduce. \\
% words: 86

\begin{figure}
\includegraphics[width=0.6\textwidth]{data/ecmwf_data}
\caption[Persistence]{Plot of the anomaly correlation coefficients of six dynamical and statistical models (S1, S2, MARKOV, CA, STAT, CLIPER) against the months of the year. This demonstrates the skill in predicting monthly Ni\~{n}o-3 index with a lead time of $+3$ months \citep{jan2005did}.}
\label{fig:ecmwf_plot}
\end{figure}

In a similar vain, \cite{jin2008current} investigated alternative coupled general circulation models (CGCMs) and analysed how they performed in ENSO prediction. With their testing they quantified that the forecast skill of individual models and MMEs depends strongly on the ENSO phase and intensity, and on the season. They found that for forecasts which start in February or May, the skill drops more sharply than predictions made in August or November. This is yet again a demonstration of the SPB and further highlights the need to understand this limitation within ENSO modelling.
% words: 90

\section{Overcoming the spring predictability barrier}

\subsection{Parameter and initial errors in the Zebiak-Cane model}

As demonstrated, the SPB is a prevalent characteristic of ENSO forecasts which exists within statistical and coupled models. Whilst it may not be feasible to entirely eliminate the barrier from predictions, it would be beneficial to reduce its effect. One potential line of inquiry is understanding the initial and parameter errors within the ZC model which leads to the SPB. Through adopting an approach of conditional nonlinear optimal perturbations (CNOP), the errors can be isolated and their impact on the SPB quantified \citep{duan2009exploring}. This method identifies the optimal error perturbations (initial or parameter) in the ZC model within given constraints through reducing evolution equations \citep{mu2010extension}. Applications of CNOP in retrospective forecasts reveals that prediction errors for a parameter error only system (CNOP-P) are small which generates a weakened SPB, on the other hand in an initial error exclusive framework (CNOP-I) significant SPBs can be produced (Fig. \ref{fig:cnop}). If both are coupled together as they would be in a realistic system, then the prediction errors are weighted by the CNOP-I results. \\
% words: 170

\begin{figure}
\includegraphics[width=0.6\textwidth]{data/cnop}
\caption[CNOP]{Plots of the mean prediction errors for 16 El Ni\~{n}o events with a lead time of 12 months for each starting month: (a) prediction errors caused by CNOP-I errors, CNOP-P errors, and their combined mode with an initial constraint of $\mid \mid u_0 \mid \mid _\alpha \leq 0.8$; (b) same analysis but with $\mid \mid u_0 \mid \mid _\alpha \leq 0.4$ \citep{yu2012does}.}
\label{fig:cnop}
\end{figure}

To reduce the effect of the CNOP-I uncertainties on the prediction results, it has been suggested that an ensemble forecast technique based on a multi-initial condition ensemble could be used in predicability studies and operational forecasts \citep{kirtman2001current}.
% words: 37

%  however as with other models, these  have issues with simulating the mean and mean annual cycle of SST, irrespective of realistic initial conditions.

\subsection{Stochastic forcing and westerly wind bursts}

It can be seen that the strength of the SPB can be reduced through considerations of the initial error values established in the prediction model. Nonetheless the origins and fundamental mechanisms of the barrier are still in debate and there is yet to be a single conclusive argument. There are two main hypotheses which attempt to justify the ENSO prediction skill during spring. The first is the ZC model which suggests that the air-sea coupling strength is the weakest during the boreal spring \citep{Zebiak:1987aa}. The second relates the seasonality of the SST anomaly variance compared with that of the stochastic noise forcing \citep{webster1992monsoon, xue1994prediction}. Whilst these theories couple and consider various components on a mostly general scale, they do not examine the potential for Westerly Wind Bursts (WWBs) to be the source of the seasonality in ENSO forecast skill. \\

A potential addition to the theory which could explain the occurrence of the SPB is by using stochastic forcing.  \\

\section{Conclusions}

To summarise, understanding the mechanisms for ENSO prediction and formation is currently limited as models such as ZC simplify the coupling between the atmosphere and ocean which limits the applicability to certain non-regular El Ni\~{n}o events. An additional challenge also arises in the form of the SPB where the predictive skill across March--May is low when compared to predictions originating from other points within the year. This barrier is evident from data and modelling, but there is yet to be a conclusive model which explains why it exists. Current research suggests that the barrier maybe related to weak climate coupling during spring or it may arise from stochastic noise forcing. However, these models fail to account for the potential for WWBs to be the source of the seasonality in ENSO forecast skill. Incorporating WWBs into prediction models suggests the SPB may be materialising due to the presence of the WWBs, thus they are important component in ENSO dynamics.
% words: 158

% Creating a sensible and logical narrative is so difficult 

% Identify the topic 
% Identify issues with the model
% Identify data and evidence that suggests model needs revising
% Suggest and defend new model or models

\clearpage

%\nocite{wang2017nino}
%\nocite{ruddiman_climate}
\bibliographystyle{agsm}
\bibliography{el_nino}

\end{document}