%%%%% Document Setup %%%%%%%%

\documentclass[12pt, onecolumn]{revtex4}    % Font size (12pt) and column number (one or two).

\usepackage[a4paper, left=2.5cm, right=2.5cm, top=2.5cm, bottom=2.5cm]{geometry}  % Defines paper size and margin length

\renewcommand{\baselinestretch}{1}     % Defines the line spacing

\usepackage{subcaption}

\usepackage[font=small, labelfont=bf]{caption}                      % Defines caption font size and caption title bolded
\captionsetup[figure]{justification=justified, singlelinecheck=off, font=footnotesize} 
\captionsetup{compatibility=false}

\usepackage{graphics,graphicx,epsfig,ulem}	% Makes sure all graphics works
\usepackage{amsmath} 						% Adds mathematical features for equations

\usepackage{etoolbox}                       % Customise date to preferred format
\makeatletter
\patchcmd{\frontmatter@RRAP@format}{(}{}{}{}
\patchcmd{\frontmatter@RRAP@format}{)}{}{}{}
\renewcommand\Dated@name{}
\makeatother

\usepackage{fancyhdr}

\pagestyle{fancy}                           % Insert header
\renewcommand{\headrulewidth}{0pt}
%\lhead{\small Jacky Cao}                        
%\rhead{\small The incomplete models for El Nino Southern Oscillations}                

\def\thesection{\arabic{section}}
\def\thesubsection{\alph{subsection}}

\def\bibsection{\section*{References}}        % Position reference section correctly
\setcitestyle{authoryear,round}
\setlength\bibhang{0.2in}
\usepackage[colorlinks]{hyperref}
\hypersetup{
    colorlinks=true,
    linkcolor=black,
    citecolor=black,    
    urlcolor=black,
}

%%%%% Document %%%%%
\begin{document}                     

\title{Modelling of the El Ni\~{n}o Southern Oscillations} 
%\date{Submitted: \today{}} \author{Jacky Cao}

\maketitle
\thispagestyle{plain} % produces page number for front 

% Introducing ENSOs and what EN and SO are individually
The El Ni\~{n}o Southern Oscillations (ENSOs) are generally known to be a composite weather phenomena originating in the Pacific Ocean which produces lasting teleconnections on the global climate system. El Ni\~{n}o can be approximately considered to be an oceanic warming event which disrupts the normal Pacific circulation at irregular intervals of 2--7 years, whilst the Southern Oscillations are an inter-annual flip between the tropical sea level pressure between the western and eastern Pacific leading to the weakening and strengthening of the easterly trade winds across the ocean. \\
% words: 83

% Introduce the idea that the link and origins of ENSOs are unknown, but the effects can be seen - give case studies
During El Ni\~{n}o years, strong trade winds are absent to transfer warm water westwards across the Pacific resulting in the water flowing back eastwards towards the American continents. This leads to warmer and wetter conditions along the western coastlines of the Americas and dryer conditions in Australia and Indonesia. This would benefit South American farmers as there would be an increase in vegetation growth but also the potential for tropical diseases such as malaria and cholera to thrive. At the same time, Australian farmers would struggle to produce crops due to the lack of rainfall. Therefore it would be beneficial to predict when an ENSO event occurs as society could then efficiently plan around it. Modern research has yet to find a reliable model even though the effects are mostly understood. \\
% words: 131

% the above is a dumb paragraph and I think I should cite previous ENSO events which have had devastating effects e.g. 1982/83 El Niño event

% Different understandings of what El Niño's are and I just want to lay the ground work and emphasise just how large a problem this is and how there really is not one reliable model for the prediction of El Niño 
At the utmost basic level there are discrepancies with the definition of ``El Ni\~{n}o''. On a national scale Australia, Peru and the USA employ different ways to classify an El Ni\~{n}o event \citep{doi:10.1175/BAMS-D-16-0009.1}. Each country experiences the effects of ENSO differently therefore each specifies alternative conditions for when an El Ni\~{n}o event is occurring. Whilst they all roughly consider oceanic and atmospheric anomalies to inform their updates, utilising one general definition would provide a better gauge of the severity of an ENSO event therefore enabling proper preparations. This lack of consensus is extended to the scientific community where there is no single identifiable model for El Ni\~{n}o due to the scale and complexity of the event \citep{1997BAMS...78.2771T}. \\
% words: 117

% don't really know what I'm trying to say with this paragraph 

% How Bjerknes set the stage for ENSO research and how that spawned a variety of research  
\cite{doi:10.1175/1520-04931969097} first theorised that a positive ocean-atmosphere feedback system leads to an El Ni\~{n}o event. An initial positive sea surface temperature (SST) anomaly in the eastern Pacific would reduce the east-west SST gradient which eventually leads to the strengthening of the Walker circulation and the production of weaker trade winds across the equatorial Pacific. In a complete ENSO theory this positive system should be counterbalanced by a negative loop which returns the Pacific to its ``normal'' (pre-ENSO) state. Whilst Bjerknes' hypothesis fails to provide a negative feedback mechanism, \cite{Zebiak:1987aa} presents a singular model which demonstrates and outlines the coupling between the atmosphere and the ocean to produce an ENSO event. The atmospheric component used is a linear Gill-type model \citep{Gill:1980aa} which describes the atmospheric response to SST anomalies, and the ocean is represented by a low-gravity model which is forced by the wind stress from the atmospheric model. This Zebiak-Cane model is used as the basic foundation for several modern ENSO oscillator theories:   \\

% HUH
This idea forms the basis of two approximate schools of thought on the theoretical origins of ENSOs \citep{wang2017nino}. The first suggests El Ni\~{n}o to be a phase of a self-sustained, unstable, and natural oscillatory mode of the coupled ocean-atmosphere system. The second, describes El Ni\~{n}o to be a stable (or damped) mode which is triggered by or interacted with random forcing or noise such as westerly wind bursts, tropical instability waves in the eastern Pacific \citep{An:2008aa}, and Madden-Julian oscillation events \citep{doi:10.1175/JAS4029.1}.  \\
% words: 116

% An oscillator is the key underlying proponent in an ENSO theory and that may be explained with Wang 
The prevalent mechanism amongst both groupings of ideas is the effect of an oscillator on the Pacific climate system. There are several conceptual models which aim to describe ENSO including those based on the coupled system, for example the delayed oscillator \citep{Suarez:1988aa, Battisti:1988aa} and the recharge-discharge oscillator \citep{Jin:1997aa}. These are generally there exists a ``unified oscillator'' \citep{Wang:2001aa} which suggests that all alternative oscillator models are a special case of itself. \\
% words: 68

% Creating a sensible and logical narrative is so difficult 

% Identify the topic
% Identify issues with the model
% Identify data and evidence that suggests model needs revising
% Suggest and defend new model or models

\newpage

%\nocite{wang2017nino}
\nocite{ruddiman_climate}
\bibliographystyle{agsm}
\bibliography{el_nino}

\end{document}