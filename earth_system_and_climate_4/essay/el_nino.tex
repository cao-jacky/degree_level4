%%%%% Document Setup %%%%%%%%

\documentclass[12pt, onecolumn]{revtex4}    % Font size (12pt) and column number (one or two).

\usepackage[a4paper, left=2.5cm, right=2.5cm, top=2.5cm, bottom=2.5cm]{geometry}  % Defines paper size and margin length

\renewcommand{\baselinestretch}{1}     % Defines the line spacing

\usepackage{subcaption}

\usepackage[font=small, labelfont=bf]{caption}                      % Defines caption font size and caption title bolded
\captionsetup[figure]{justification=justified, singlelinecheck=off, font=footnotesize} 
\captionsetup{compatibility=false}

\usepackage{graphics,graphicx,epsfig,ulem}	% Makes sure all graphics works
\usepackage{amsmath} 						% Adds mathematical features for equations

\usepackage{etoolbox}                       % Customise date to preferred format
\makeatletter
\patchcmd{\frontmatter@RRAP@format}{(}{}{}{}
\patchcmd{\frontmatter@RRAP@format}{)}{}{}{}
\renewcommand\Dated@name{}
\makeatother

\usepackage{fancyhdr}

\pagestyle{fancy}                           % Insert header
\renewcommand{\headrulewidth}{0pt}
%\lhead{\small Jacky Cao}                        
%\rhead{\small The incomplete models for El Nino Southern Oscillations}                

\def\thesection{\arabic{section}}
\def\thesubsection{\alph{subsection}}

\def\bibsection{\section*{References}}        % Position reference section correctly
\setcitestyle{authoryear,round}
\setlength\bibhang{0.2in}
\usepackage[colorlinks]{hyperref}
\hypersetup{
    colorlinks=true,
    linkcolor=black,
    citecolor=black,    
    urlcolor=black,
}

%%%%% Document %%%%%
\begin{document}                     

\title{Modelling of the El Ni\~{n}o Southern Oscillations} 
%\date{Submitted: \today{}} \author{Jacky Cao}

\maketitle
\thispagestyle{plain} % produces page number for front 

% Introducing ENSOs and why they are important
The El Ni\~{n}o Southern Oscillations (ENSOs) are a composite weather phenomena originating in the Pacific Ocean which produce lasting effects and teleconnections on the global climate system. El Ni\~{n}o can be considered to be an oceanic warming event which occurs every few years. Whilst the Southern Oscillation is an interannual flip between the tropical sea level pressure between the western and eastern Pacific, in essence it is the weakening and strengthening of the easterly trade winds across the Pacific. \cite{doi:10.1175/1520-04931969097} provided a link between the two effects and described them to be different aspects of the same phenomena. \\

% talk about how there isn't one true El Niño definition 

% Identify the topic
% Identify issues with the model
% Identify data and evidence that suggests model needs revising
% Suggest and defend new model or models

\newpage

\nocite{wang2017nino}
\bibliographystyle{agsm}
\bibliography{el_nino}

\end{document}