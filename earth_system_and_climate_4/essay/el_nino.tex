%%%%% Document Setup %%%%%%%%

\documentclass[12pt, onecolumn]{revtex4}    % Font size (12pt) and column number (one or two).

\usepackage[a4paper, left=2.5cm, right=2.5cm, top=2.5cm, bottom=2.5cm]{geometry}  % Defines paper size and margin length

\renewcommand{\baselinestretch}{1}     % Defines the line spacing

\usepackage{subcaption}

\usepackage[font=small, labelfont=bf]{caption}                      % Defines caption font size and caption title bolded
\captionsetup[figure]{justification=justified, singlelinecheck=off, font=footnotesize} 
\captionsetup{compatibility=false}

\usepackage{graphics,graphicx,epsfig,ulem}	% Makes sure all graphics works
\usepackage{amsmath} 						% Adds mathematical features for equations

\usepackage{etoolbox}                       % Customise date to preferred format
\makeatletter
\patchcmd{\frontmatter@RRAP@format}{(}{}{}{}
\patchcmd{\frontmatter@RRAP@format}{)}{}{}{}
\renewcommand\Dated@name{}
\makeatother

\usepackage{fancyhdr}

\pagestyle{fancy}                           % Insert header
\renewcommand{\headrulewidth}{0pt}
%\lhead{\small Jacky Cao}                        
%\rhead{\small The incomplete models for El Nino Southern Oscillations}                

\def\thesection{\arabic{section}}
\def\thesubsection{\alph{subsection}}

\def\bibsection{\section*{References}}        % Position reference section correctly
\setcitestyle{authoryear,round}
\setlength\bibhang{0.2in}
\usepackage[colorlinks]{hyperref}
\hypersetup{
    colorlinks=true,
    linkcolor=black,
    citecolor=black,    
    urlcolor=black,
}

%%%%% Document %%%%%
\begin{document}                     

\title{Modelling of the El Ni\~{n}o Southern Oscillations} 
%\date{Submitted: \today{}} \author{Jacky Cao}

\maketitle
\thispagestyle{plain} % produces page number for front 

% Introducing ENSOs and what EN and SO are individually
The El Ni\~{n}o Southern Oscillations (ENSOs) are a composite weather phenomena originating in the Pacific Ocean which produce lasting teleconnections on the global climate system. El Ni\~{n}o can be considered to be an oceanic warming event which disrupts the normal Pacific circulation at irregular intervals of 2--7 years, whilst the Southern Oscillations are an inter-annual flip between the tropical sea level pressure between the western and eastern Pacific leading to the weakening and strengthening of the easterly trade winds across the Pacific. \\
% words: 82

% Introduce the idea that the link and origins of ENSOs are unknown, but the effects can be seen - give case studies
During El Ni\~{n}o years, strong trade winds are absent to transfer warm water westwards across the Pacific, the result is that the water flows back eastwards towards the American continents. This leads to warmer and wetter conditions along the western coastlines of the Americas and dryer conditions in Australia and Indonesia. South American farmers would benefit as there would be increased vegetation growth but at the same time potentially suffer due to the breeding and spreading of tropical diseases such as malaria and cholera. On the other hand, Australian farmers would struggle due to the lack of rainfall. Whilst the effects of the coupled ENSOs can be approximately understood, modern research has yet to find a reliable model which can predict when and how they occur. \\
% words: 107

% Different understandings of what El Niño's are and I just want to lay the ground work and emphasise just how large a problem this is and how there really is not one reliable model for the prediction of El Niño 
This discussion begins with how El Ni\~{n}o is defined and the issues surrounding this. On a national scale Australia, Peru and the USA use different ways to classify an El Ni\~{n}o event \citep{doi:10.1175/BAMS-D-16-0009.1}. Each country experiences the effects of ENSO differently therefore each specifies alternative conditions for when an El Ni\~{n}o event is occurring. Whilst they all roughly consider oceanic and atmospheric anomalies to inform their updates, utilising one general definition would provide a better gauge of the severity of an ENSO event therefore enabling proper preparations. This lack of consensus can also be extended to the scientific community where there is no single identifiable model for El Ni\~{n}o nor even a single definition \citep{1997BAMS...78.2771T}. \\
% words: 115

% How Bjerknes set the stage for ENSO research and how that spawned a variety of research  
\cite{doi:10.1175/1520-04931969097} initially identified that El Ni\~{n}o and Southern Oscillations are different aspects of the same phenomena, they proposed that the cause of ENSO is related to a positive ocean-atmosphere feedback loop involving the Walker circulation. This idea forms the basis of two approximate schools of thought on the theoretical origins of ENSOs \citep{wang2017nino}. The first suggests El Ni\~{n}o to be a phase of a self-sustained, unstable, and natural oscillatory mode of the coupled ocean-atmosphere system. The second, describes El Ni\~{n}o to be a stable (or damped) mode which is triggered by or interacted with random forcing or noise such as westerly wind bursts, tropical instability waves in the eastern Pacific \citep{An:2008aa}, and Madden-Julian oscillation events \citep{doi:10.1175/JAS4029.1}. \\
% words: 116

% There is not one true El Niño model, there are multiple competing theories
Exploring individual theories leads to a greater sense of uncertainty as competing hypotheses try to describe different aspects of the ENSO phenomenon.  As of yet there is no one true ``grand unifying El Ni\~{n}o theory'' however the separate theories may provide various possibilities to enable the production of one.  \\
% words: 49

% \cite{Takahashi2018} and \cite{Hameed:2018aa} attempt to describe El Ni\~{n}o events which have stronger than normal strengths.

% Mathematical and statistical analysis would be the ideal method to verify a single theory 


% ENSOs and global warming

% Identify the topic
% Identify issues with the model
% Identify data and evidence that suggests model needs revising
% Suggest and defend new model or models

\newpage

%\nocite{wang2017nino}
\nocite{ruddiman_climate}
\bibliographystyle{agsm}
\bibliography{el_nino}

\end{document}