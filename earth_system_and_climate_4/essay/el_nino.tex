%%%%% Document Setup %%%%%%%%

\documentclass[12pt, onecolumn]{revtex4}    % Font size (12pt) and column number (one or two).

\usepackage[a4paper, left=2.5cm, right=2.5cm, top=2.5cm, bottom=2.5cm]{geometry}  % Defines paper size and margin length

\renewcommand{\baselinestretch}{1}     % Defines the line spacing

\usepackage{subcaption}

\usepackage[font=small, labelfont=bf]{caption}                      % Defines caption font size and caption title bolded
\captionsetup[figure]{justification=justified, singlelinecheck=off, font=footnotesize} 
\captionsetup{compatibility=false}

\usepackage{graphics,graphicx,epsfig,ulem}	% Makes sure all graphics works
\usepackage{amsmath} 						% Adds mathematical features for equations

\usepackage{etoolbox}                       % Customise date to preferred format
\makeatletter
\patchcmd{\frontmatter@RRAP@format}{(}{}{}{}
\patchcmd{\frontmatter@RRAP@format}{)}{}{}{}
\renewcommand\Dated@name{}
\makeatother

\usepackage{fancyhdr}

\pagestyle{fancy}                           % Insert header
\renewcommand{\headrulewidth}{0pt}
%\lhead{\small Jacky Cao}                        
%\rhead{\small The incomplete models for El Nino Southern Oscillations}                

\def\thesection{\arabic{section}}
\def\thesubsection{\alph{subsection}}

\def\bibsection{\section*{References}}        % Position reference section correctly
\setcitestyle{authoryear,round}
\setlength\bibhang{0.2in}
\usepackage[colorlinks]{hyperref}
\hypersetup{
    colorlinks=true,
    linkcolor=black,
    citecolor=black,    
    urlcolor=black,
}

%%%%% Document %%%%%
\begin{document}                     

\title{Theoretical predictions of the El Ni\~{n}o Southern Oscillations with the Zebiak-Cane model} 
%\date{Submitted: \today{}} \author{Jacky Cao}

\maketitle
\thispagestyle{plain} % produces page number for front 

\section{Introduction}
% Introducing ENSOs and what EN and SO are individually and then setting the problem from the offset 
The El Ni\~{n}o Southern Oscillations (ENSOs) are generally known as a composite weather phenomena originating in the Pacific Ocean producing lasting teleconnections on the global climate system. The El Ni\~{n}o component of ENSO can be approximately considered to be an oceanic warming event which disrupts the normal Pacific circulation at irregular intervals of 2--7 years, whilst the Southern Oscillations are an inter-annual flip of the tropical sea level pressure between the western and eastern Pacific leading to the weakening and strengthening of the easterly trade winds across the ocean. To produce a conclusive theory for ENSOs one must be able to describe and understand the complete underlying mechanisms. One such hypothesis has yet to arise, however various attempts have been made to comprehend individual components and effects. \\
% words: 127

% How Bjerknes, Zebiak and Cane set the stage for ENSO research 
\cite{doi:10.1175/1520-04931969097} first theorised that a positive ocean-atmosphere feedback system would result in an El Ni\~{n}o event. An initial positive sea surface temperature (SST) anomaly in the eastern Pacific would reduce the east-west SST gradient which leads to the strengthening of the Walker circulation and thus the production of weaker trade winds across the equatorial Pacific. In a complete ENSO theory this positive system would be counterbalanced by a negative loop which returns the Pacific to its ``normal'' (pre-ENSO) state. Whilst Bjerknes' hypothesis fails to provide a negative feedback mechanism, \cite{Zebiak:1987aa} presented a model which demonstrated and outlined the coupling between the atmosphere and the ocean to produce an ENSO event. The atmospheric component used was a linear Gill-type model \citep{Gill:1980aa} which describes the atmosphere's response to SST anomalies, and the ocean represented by a low-gravity system which is forced by the wind stress from the atmospheric constituent. With their model they were able to produce certain features observed during ENSO events such as equatorial westerly wind anomalies in the central Pacific and large SST anomalies in the eastern Pacific. Despite this success, they recognised their limited ability in simulating the real complete system as detailed comparisons with observational data would reveal discrepancies in their atmospheric and oceanic simulations. They suggest that more sophisticated models would have to be produced to correctly simulate the real ENSO cycle. \\
% words: 226 

\section{Improving the Zebiak-Cane model}

For several modern ENSO oscillator theories, the Zebiak-Cane (ZC) model is used as a basic foundation: the delayed oscillator \citep{Suarez:1988aa, Battisti:1988aa}, the recharge oscillator \citep{Jin:1997aa}, the western Pacific oscillator \citep{Weisberg:1997aa, wang1999effects}, and the advective-reflective oscillator \citep{Picaut663}. These variants of Zebiak and Cane's work provide adjustments to the atmospheric and oceanic components of the ZC model which allows for improvements in the accuracy and precision in the prediction of ENSO events. \\

It has 

% Creating a sensible and logical narrative is so difficult 

% Identify the topic
% Identify issues with the model
% Identify data and evidence that suggests model needs revising
% Suggest and defend new model or models

\newpage

%\nocite{wang2017nino}
\nocite{ruddiman_climate}
\bibliographystyle{agsm}
\bibliography{el_nino}

\end{document}