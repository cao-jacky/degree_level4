%%%%% Document Setup %%%%%%%%

\documentclass[12pt, onecolumn]{revtex4}    % Font size (12pt) and column number (one or two).

\usepackage[a4paper, left=2.5cm, right=2.5cm, top=2.5cm, bottom=2.5cm]{geometry}  % Defines paper size and margin length

\renewcommand{\baselinestretch}{1}     % Defines the line spacing

\usepackage{subcaption}

\usepackage[font=small, labelfont=bf]{caption}                      % Defines caption font size and caption title bolded
\captionsetup[figure]{justification=justified, singlelinecheck=off, font=footnotesize} 
\captionsetup{compatibility=false}

\usepackage{graphics,graphicx,epsfig,ulem}	% Makes sure all graphics works
\usepackage{amsmath} 						% Adds mathematical features for equations

\usepackage{etoolbox}                       % Customise date to preferred format
\makeatletter
\patchcmd{\frontmatter@RRAP@format}{(}{}{}{}
\patchcmd{\frontmatter@RRAP@format}{)}{}{}{}
\renewcommand\Dated@name{}
\makeatother

\usepackage{fancyhdr}

\pagestyle{fancy}                           % Insert header
\renewcommand{\headrulewidth}{0pt}
%\lhead{\small Jacky Cao}                        
%\rhead{\small The incomplete models for El Nino Southern Oscillations}                

\def\thesection{\arabic{section}}
\def\thesubsection{\alph{subsection}}

\def\bibsection{\section*{References}}        % Position reference section correctly
\setcitestyle{authoryear,round}
\setlength\bibhang{0.2in}
\usepackage[colorlinks]{hyperref}
\hypersetup{
    colorlinks=true,
    linkcolor=black,
    citecolor=black,    
    urlcolor=black,
}

%%%%% Document %%%%%
\begin{document}                     

\title{Modelling the El Ni\~{n}o Southern Oscillations through the work of Zebiak-Cane} 
%\date{Submitted: \today{}} \author{Jacky Cao}

\maketitle
\thispagestyle{plain} % produces page number for front 

% Introducing ENSOs and what EN and SO are individually and then setting the problem from the offset 
The El Ni\~{n}o Southern Oscillations (ENSOs) are generally known as a composite weather phenomena originating in the Pacific Ocean which produces lasting teleconnections on the global climate system. El Ni\~{n}o can be approximately considered to be an oceanic warming event which disrupts the normal Pacific circulation at irregular intervals of 2--7 years, whilst the Southern Oscillations are an inter-annual flip between the tropical sea level pressure between the western and eastern Pacific leading to the weakening and strengthening of the easterly trade winds across the ocean. Whilst the effects of ENSOs are mostly well understood, there is yet to be one conclusive theory which can describes the underlying mechanisms. \\
% words: 109

% How Bjerknes set the stage for ENSO research and how that spawned a variety of research  
\cite{doi:10.1175/1520-04931969097} first theorised that a positive ocean-atmosphere feedback system results in an El Ni\~{n}o event. An initial positive sea surface temperature (SST) anomaly in the eastern Pacific would reduce the east-west SST gradient which leads to the strengthening of the Walker circulation and the production of weaker trade winds across the equatorial Pacific. In a complete ENSO theory this positive system should be counterbalanced by a negative loop which returns the Pacific to its ``normal'' (pre-ENSO) state. Whilst Bjerknes' hypothesis fails to provide a negative feedback mechanism, \cite{Zebiak:1987aa} presents a singular model which demonstrates and outlines the coupling between the atmosphere and the ocean to produce an ENSO event. The atmospheric component is a linear Gill-type model \citep{Gill:1980aa} which describes the atmospheric response to SST anomalies, and the ocean is represented by a low-gravity model which is forced by the wind stress from the atmospheric component. The Zebiak-Cane (ZC) model is commonly used as the basic foundation for several modern ENSO oscillator theories: the delayed oscillator \citep{Suarez:1988aa, Battisti:1988aa}, the recharge oscillator \citep{Jin:1997aa}, the western Pacific oscillator \citep{Weisberg:1997aa, wang1999effects}, and the advective-reflective oscillator \citep{Picaut663}. \\
% words: 183

% the importance of the ZC model and how it has been worked upon to improve various aspects
These variants of Zebiak and Cane's work provide adjustments to the atmospheric and oceanic components of the ZC model, allowing for improvements in the accuracy and precision in the prediction of ENSO events. \\


% Creating a sensible and logical narrative is so difficult 

% Identify the topic
% Identify issues with the model
% Identify data and evidence that suggests model needs revising
% Suggest and defend new model or models

\newpage

%\nocite{wang2017nino}
\nocite{ruddiman_climate}
\bibliographystyle{agsm}
\bibliography{el_nino}

\end{document}