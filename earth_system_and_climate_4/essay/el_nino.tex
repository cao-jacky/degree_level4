%%%%% Document Setup %%%%%%%%

\documentclass[12pt, onecolumn]{revtex4}    % Font size (12pt) and column number (one or two).

\usepackage[a4paper, left=2.5cm, right=2.5cm, top=2.5cm, bottom=2.5cm]{geometry}  % Defines paper size and margin length

\renewcommand{\baselinestretch}{1}     % Defines the line spacing

\usepackage{subcaption}

\usepackage[font=small, labelfont=bf]{caption}                      % Defines caption font size and caption title bolded
\captionsetup[figure]{justification=justified, singlelinecheck=off, font=footnotesize} 
\captionsetup{compatibility=false}

\usepackage{graphics,graphicx,epsfig,ulem}	% Makes sure all graphics works
\usepackage{amsmath} 						% Adds mathematical features for equations

\usepackage{etoolbox}                       % Customise date to preferred format
\makeatletter
\patchcmd{\frontmatter@RRAP@format}{(}{}{}{}
\patchcmd{\frontmatter@RRAP@format}{)}{}{}{}
\renewcommand\Dated@name{}
\makeatother

\usepackage{fancyhdr}

\pagestyle{fancy}                           % Insert header
\renewcommand{\headrulewidth}{0pt}
%\lhead{\small Jacky Cao}                        
%\rhead{\small The incomplete models for El Nino Southern Oscillations}                

\def\thesection{\arabic{section}}
\def\thesubsection{\alph{subsection}}

\def\bibsection{\section*{References}}        % Position reference section correctly
\setcitestyle{authoryear,round}
\setlength\bibhang{0.2in}
\usepackage[colorlinks]{hyperref}
\hypersetup{
    colorlinks=true,
    linkcolor=black,
    citecolor=black,    
    urlcolor=black,
}

%%%%% Document %%%%%
\begin{document}                     

\title{Modelling of the El Ni\~{n}o Southern Oscillations} 
%\date{Submitted: \today{}} \author{Jacky Cao}

\maketitle
\thispagestyle{plain} % produces page number for front 

% Introducing ENSOs and what EN and SO are individually
The El Ni\~{n}o Southern Oscillations (ENSOs) are a composite weather phenomena originating in the Pacific Ocean which produce lasting teleconnections on the global climate system. El Ni\~{n}o can be considered to be an oceanic warming event which disrupts the normal Pacific circulation at irregular intervals of 2--7 years. Whilst the Southern Oscillation are an inter-annual flip between the tropical sea level pressure between the western and eastern Pacific leading to the weakening and strengthening of the easterly trade winds across the Pacific. \\
% words: 82

% Introduce the idea that the link and origins of ENSOs are unknown, but the effects can be seen - give case studies
During El Ni\~{n}o years, strong trade winds are absent to transfer warm water westwards across the Pacific, the result is that the water flows back eastwards towards the American continents. This leads to warmer and wetter conditions along the western coastlines of the Americas and dryer conditions in Australia and Indonesia. The positives for this include better vegetation growth for South American farmers and negatives would be the breeding and spreading of tropical diseases such as malaria and cholera. Whilst the effects of the coupled ENSOs can be approximately understood, modern research has yet to find a reliable model which can predict when and how they occur. \\
% words: 107

% Bjerknes and the initial ideas for how ENSOs are produced 
\cite{doi:10.1175/1520-04931969097} identified El Ni\~{n}o and the Southern Oscillations are connected through being different aspects of the same phenomena. \\

% talk about how there isn't one true El Niño definition 

% Identify the topic
% Identify issues with the model
% Identify data and evidence that suggests model needs revising
% Suggest and defend new model or models

\newpage

\nocite{wang2017nino}
\nocite{ruddiman_climate}
\bibliographystyle{agsm}
\bibliography{el_nino}

\end{document}