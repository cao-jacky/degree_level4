%%%%% Document Setup %%%%%%%%

\documentclass[12pt, onecolumn]{revtex4}    % Font size (12pt) and column number (one or two).

\usepackage[a4paper, left=2.5cm, right=2.5cm, top=2.5cm, bottom=2.5cm]{geometry}  % Defines paper size and margin length

\renewcommand{\baselinestretch}{1}     % Defines the line spacing

\usepackage{subcaption}

\usepackage[font=small, labelfont=bf]{caption}                      % Defines caption font size and caption title bolded
\captionsetup[figure]{justification=justified, singlelinecheck=off, font=footnotesize} 
\captionsetup{compatibility=false}

\usepackage{graphics,graphicx,epsfig,ulem}	% Makes sure all graphics works
\usepackage{amsmath} 						% Adds mathematical features for equations

\usepackage{etoolbox}                       % Customise date to preferred format
\makeatletter
\patchcmd{\frontmatter@RRAP@format}{(}{}{}{}
\patchcmd{\frontmatter@RRAP@format}{)}{}{}{}
\renewcommand\Dated@name{}
\makeatother

\usepackage{fancyhdr}

\pagestyle{fancy}                           % Insert header
\renewcommand{\headrulewidth}{0pt}
%\lhead{\small Jacky Cao}                        
\rhead{\small }                

\def\thesection{\arabic{section}}
\def\thesubsection{\alph{subsection}}

\def\bibsection{\section*{References}}        % Position reference section correctly
\setcitestyle{authoryear,round}
\setlength\bibhang{0.2in}
\usepackage[colorlinks]{hyperref}
\hypersetup{
    colorlinks=true,
    linkcolor=black,
    citecolor=black,    
    urlcolor=black,
}

\usepackage{tabularx}

%%%%% Document %%%%%
\begin{document}                     

\title{Predicting El Ni\~{n}o Southern Oscillations and the `spring predictability barrier'} 
%\date{Submitted: \today{}} \author{Jacky Cao}

\maketitle
\thispagestyle{plain} % produces page number for front 

\section{Introduction}
% Introducing ENSOs and what EN and SO are individually and then setting the problem from the offset 
The El Ni\~{n}o Southern Oscillations (ENSOs) are generally known as a composite weather phenomena originating in the Pacific Ocean producing lasting teleconnections on the global climate system. The El Ni\~{n}o component of ENSO can be approximately considered to be an oceanic warming event which disrupts the normal Pacific circulation at irregular intervals of 2--7 years, whilst the Southern Oscillations are an inter-annual flip of the tropical sea level pressure between the western and eastern Pacific leading to the weakening and strengthening of the easterly trade winds across the ocean. To produce a conclusive theory for ENSOs one must be able to describe and understand the complete underlying mechanisms. One such hypothesis has yet to arise, however various attempts have been made to comprehend individual components and effects. \\
% words: 127

% How Bjerknes, Zebiak and Cane set the stage for ENSO research 
\cite{doi:10.1175/1520-04931969097} first theorised that a positive ocean-atmosphere feedback system would result in an El Ni\~{n}o event. An initial positive sea surface temperature (SST) anomaly in the eastern Pacific would reduce the east-west SST gradient which leads to the strengthening of the Walker circulation and thus the production of weaker trade winds across the equatorial Pacific. In a complete ENSO theory this positive system would be counterbalanced by a negative loop which returns the Pacific to its ``normal'' (pre-ENSO) state. Whilst Bjerknes' hypothesis fails to provide a negative feedback mechanism, \cite{Zebiak:1987aa} presented a model which demonstrated and outlined the coupling between the atmosphere and the ocean to produce an ENSO event. The atmospheric component used was a linear Gill-type model \citep{Gill:1980aa} which describes the atmosphere's response to SST anomalies, and the ocean represented by a low-gravity system which is forced by the wind stress from the atmospheric constituent. \\
% words: 147

With their model they were able to replicate features observed during ENSO events such as equatorial westerly wind anomalies in the central Pacific and large SST anomalies in the eastern Pacific, on top of that they were able to predict the onset of the 1986--1987 and 1991--1992 ENSO events. Despite this success, they recognised their limited ability in simulating the real complete system as detailed comparisons with observational data would reveal discrepancies in their atmospheric and oceanic simulations. Furthermore, the short warm episodes in 1993 and 1994 would be missed in the predictions made with the Zebiak-Cane (ZC) model. This therefore requires more sophisticated models to better describe and forecast ENSO events.
% words: 112

\section{Prediction limitations with the Zebiak-Cane model}

The prediction of ENSO events is particularly difficult as there are generally two types of El Ni\~{n}o events to account for. The first are canonical events which generally develop along the South American coast and then they propagate westwards across the Pacific, ``Eastern-Pacific'' events \citep{rasmusson1982variations}. The second type of events have warm SST mostly centred in the central Pacific which do not propagate, ``Central-Pacific'' events \citep{ashok2007nino}. In an attempt to test whether the ZC model can predict either types of events, it was demonstrated that the model tended to do well whilst simulating Eastern-Pacific (EP) events and functioned badly when portraying Central-Pacific (CP) events \citep{duan2013behaviors}. This indicates that the ZC model may just contain the physics to explain EP events and that alterations would be required to additionally account for CP events. \\
% words: 132

On top of that, additional challenges for ENSO forecasting arises due to the coupling between the ocean and atmospheric systems being nonlinear and complex. Within the ZC model this relationship was constrained as a result of Zebiak and Cane's assumptions and parameter choices when building their model. For example, experiments would show that changes in certain values would amount to an increase (decrease) in the strength of the coupling between atmosphere and ocean so the amplitudes and periods of the ENSO cycle would increase (decrease) as a result. \\

% how is that actually bad/why is that actually a problem? 

In an improved model, theoretical modifications of the individual atmospheric and oceanic components plus of their inter-relationship would be required to provide a better fit with the observational data. Several modern ENSO oscillator theories (Table \ref{table:enso_oscillators}) employ the ZC model as a basic foundation model whilst making adjustments to the atmospheric or oceanic components. 
% words: 142

\begin{table}[htbp]
\renewcommand{\arraystretch}{1.0}
%\begin{tabular}{c@{\hskip 20pt}c} 
\begin{tabular}{|p{7cm}|p{9cm}|}
 \hline
 \textbf{Theory} & \textbf{Main component(s)} \\ [0.5ex] 
 \hline
 The Delayed Oscillator \citep{Suarez:1988aa, Battisti:1988aa} & Considers the effects of equatorially trapped oceanic wave propagation. \\
 \hline
 The Recharge Oscillator \citep{Jin:1997aa} & Considers the buildup of warm water in the western Pacific as a precondition to the development of El Ni\~{n}o. \\
 \hline
 The Western Pacific Oscillator \citep{Weisberg:1997aa, wang1999effects} & Considers the role of the western Pacific and off-equatorial SST SST anomalies in the western Pacific. \\
 \hline
 The Advective-Reflective Oscillator \citep{Picaut663} & Considers the importance of the positive feedback of zonal currents that advect the western Pacific warm pool towards the east during El Ni\~{n}o.  \\
 \hline
 The Unified Oscillator \citep{wang2001unified} & Considers dynamics and thermodynamics of a coupled ocean-atmosphere system which is similar to Zebiak-Cane. \\
 \hline
\end{tabular}
\captionof{table}[ENSO Oscillators]{Various ENSO oscillator theories and their main differences from the ZC model.}
\label{table:enso_oscillators}
\end{table}

The existence of multiple hypotheses highlights the intrinsic complexities of ENSO events and thus warrants further research into the predictability of ENSOs. One issue which arises within the data and models of El Ni\~{n}o \\

\section{Improving the Zebiak-Cane model}


% Creating a sensible and logical narrative is so difficult 

% Identify the topic
% Identify issues with the model
% Identify data and evidence that suggests model needs revising
% Suggest and defend new model or models

\newpage

%\nocite{wang2017nino}
%\nocite{ruddiman_climate}
\bibliographystyle{agsm}
\bibliography{el_nino}

\end{document}