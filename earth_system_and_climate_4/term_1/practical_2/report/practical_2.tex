%%%%% Document Setup %%%%%%%%

\documentclass[12pt, onecolumn]{revtex4}    % Font size (12pt) and column number (one or two).

\usepackage[a4paper, left=2.5cm, right=2.5cm, top=2.5cm, bottom=2.5cm]{geometry}  % Defines paper size and margin length

\renewcommand{\baselinestretch}{1}     % Defines the line spacing

\usepackage{subcaption}

\usepackage[font=small, labelfont=bf]{caption}                      % Defines caption font size and caption title bolded
\captionsetup[figure]{justification=justified, singlelinecheck, font=small} 
\captionsetup[table]{justification=justified, singlelinecheck=off, font=small} 
\captionsetup{compatibility=false}

\usepackage{graphics,graphicx,epsfig,ulem}	% Makes sure all graphics works
\usepackage{amsmath} 						% Adds mathematical features for equations

\usepackage{fancyhdr}

\usepackage{textcomp}

\usepackage{tabularx}

\def\thesection{\arabic{section}}
\def\thesubsection{\alph{subsection}}

\def\bibsection{\section*{References}}        % Position reference section correctly

%%%%% Document %%%%%
\begin{document}                     

\title{Testing the Milankovitch-Croll hypothesis using $\delta^{18}$O foram data} 
\maketitle
%\thispagestyle{plain} % produces page number for front page

\vspace{-4ex}

The Milankovitch-Croll hypothesis suggests that changes in the Earth's orbit around the Sun leads to changes in the Earth's planetary climate through fluctuations in solar insolation \cite{ruddiman_climate}. We can access and view this change of climate through analysing proxy data. By using deep ocean sediment cores and Earth orbital data, we can explore the hypothesis and it's validity through performing various data analysis techniques. \\

We begin with the importance of sediment cores. Their usefulness arises from the fact that they contain a trace for measuring the $\delta^{18}$O ratio across periods of Earth's history \cite{droxler_climate}. The change in this value can provide us with an indication of how the global temperature has been fluctuating across eras and it can tell us about global ice volume. \\

Defined from laboratory experiments, for a $\sim 5^{\circ}\mathrm{C}$ temperature increase there is an $\sim 1$\textperthousand\ decrease in the $\delta^{18}$O ratio of a foram shell. As ice-sheets melt, more $^{16}$O is released into the oceans, thus reducing the ratio. Looking at Figure \ref{fig:foram_data} we can see the sediment core data for two periods from Earth's history. In the more recent age the amplitude of the $\delta^{18}$O varies more considerably than from 4 Myr to 5 Myr, suggesting that close to present times the temperature has shifted dramatically leading to longer periods of cooling. With a cooler climate, there would be a larger amount of ice on the planet. Comparing with the older $\delta^{18}$O data, we see that it still features the sinusoidal-like trend, albeit with a smaller amplitude. This suggests that the glacial-interglacial events are a normal part of Earth's planetary climate cycle and could be related to the cyclical Earth-Sun orbital relationship. \\

% not a fan of the above paragraph, it's a bit convoluted
% TODO discuss changes in global ice volume, particularly in the last 2-3 million years

In looking at our astrophysical data there are three concepts to consider, namely eccentricity, precession, and obliquity. The former describes the ellipticity of the Earth's orbit around the Sun \cite{carroll_astro}, it's effect on insolation is not entirely independent as the eccentricity also depends on the orbital angle between the solstices and the position of perihelion/aphelion. We can produce a precessional index value with $\chi= \varepsilon \sin{\omega}$, where $\varepsilon$ is the eccentricity and $\omega$ is the angle. Finally we can define obliquity as the value which describes the degree of tilt of the Earth's axis. \\

With these definitions in mind, we can proceed to evaluate and verify that our orbital data agrees with literature by applying two statistical analysis techniques. The first of which is \textit{wavelet analysis} (WA), by using the WA tool in PAST3 \cite{past3} we can obtain heatmaps which contain the dominant frequencies from a dataset. As shown in Figure \ref{fig:wa_orbital_data} (a)-(c) and summarised in Table \ref{table:final_results}, we can compare the obtained values to the literature wavelengths and find that there is an average $\sim 96 \%$ agreement. Our second tool is REDFIT analysis, an example of method applied to the benthic foram data can be seen in Figure \ref{fig:d18o_redfit}, where frequencies can be matched up to peaks which have a height greater than the $95\%$ confidence interval  \\

If we are able to so that the tools can be used on the $\delta^{18}$O foram data as well. \\

Comparison with the ``true'' wavelengths, from Table \ref{table:final_results} 

\newpage

%\twocolumngrid

\bibliographystyle{unsrt}
\bibliography{practical_2}

\newpage

\section*{Appendices}
\begin{figure}[!h]
\begin{center}
\includegraphics[width=11cm]{figures/foram_data}
\caption[]{Plots of the marine benthic foram $\delta^{18}$O data against age, with increasing time to the right. Both data sets have been taken out of a larger set which spans from 0 Myr to 6 Myr. As we move closer towards the present day, the periodicity and amplitude is larger which implies more extreme temperature variations so longer glacial-interglacials and a larger change in the global ice volume.}
\vspace{-3ex}
\label{fig:foram_data}
\end{center}
\end{figure}

\begin{figure}[!h]
\begin{center}
\includegraphics[width=16cm]{figures/wa_orbital_data}
\caption[]{The eccentricity, obliquity and perihelion longitude (precession) orbital data which has been processed with the wavelet analysis tool in PAST3. The periodicities from the data can be obtained by considering the ``hottest'' areas of each of the heat maps (i.e. the red/orange areas). Values for the wavelength can be found with $2^y \times P$, where $y$ is the y-axis height of the hot region, and $P$ is the period of time between the data points. In Table \ref{table:final_results} we summarise the wavelengths which correspond to the dashed-line heights.}
\vspace{-3ex}
\label{fig:wa_orbital_data}
\end{center}
\end{figure}

\begin{figure}[!h]
\begin{center}
\includegraphics[width=11cm]{figures/wa_d18O.pdf}
\caption[]{The result from applying the PAST3 wavelet analysis tool to the $\delta^{18}$O benthic foram data, the wavelengths found are summarised in Table \ref{table:final_results}. The results are similar to those found in the orbital data, which could imply that orbital forcing does have an influence on glacial-interglacial climate.}
\vspace{-3ex}
\label{fig:wa_d18o}
\end{center}
\end{figure}

\begin{figure}[!h]
\begin{center}
\includegraphics[width=11cm]{figures/d18O_redfit}
\caption[]{The benthic foram $\delta^{18}$O data analysed using the REDFIT spectral analysis tool in PAST3. Three dominant signals can be seen as coloured in green, they are defined as the main signals as they have a height greater than the $95\%$ confidence interval (the red curve). The same results can be found in Table \ref{table:final_results}, and comparing to the ``true'' wavelengths we also find similarities which further strengthens the possibility that the Milankovitch-Croll hypothesis has some validity.}
\vspace{-3ex}
\label{fig:d18o_redfit}
\end{center}
\end{figure}

\begin{table}[h!]
\centering
\begin{tabular}{c@{\hskip 20pt}c@{\hskip 20pt}c@{\hskip 20pt}c} 
 \hline
  & \textbf{``True'' (ka)} &\textbf{REDFIT (ka)} & \textbf{WA (ka)} \\ [0.5ex] 
 Eccentricity & 100, 413 & 95.2, 128.2, 416.7 & 89.1, 100.4,  401.7\\
 Perihelion Longitude & 23, 100 & 23.8, 22.2 & 21.1 \\
 Obliquity & 41 & 41.7 & 39.4 \\
 $\delta^{18}$O & & 23.7, 41.1, 96.3  & 38.9, 96.0, 271.5, \\
 & & & 945.5, 1247.6 \\
 \hline
\end{tabular}
\caption{Table showing different sets of wavelength data: the approximate correct wavelengths for the periodicity of the orbital features \cite{campisano_milankovitch}, the results of the REDFIT spectral analysis and Wavelet Analysis (WA) on the orbital and benthic foram data. We find that as tools both REDFIT and WA are able to obtain similar periodicities which then agree with the literature values. }
\vspace{-0.5em}
\label{table:final_results}
\end{table}

\end{document}