%%%%% Document Setup %%%%%%%%

\documentclass[12pt, onecolumn]{revtex4}    % Font size (12pt) and column number (one or two).

\usepackage[a4paper, left=2.5cm, right=2.5cm, top=2.5cm, bottom=2.5cm]{geometry}  % Defines paper size and margin length

\renewcommand{\baselinestretch}{1}     % Defines the line spacing

\usepackage{subcaption}

\usepackage[font=small, labelfont=bf]{caption}                      % Defines caption font size and caption title bolded
\captionsetup[figure]{justification=justified, singlelinecheck=off, font=footnotesize} 
\captionsetup{compatibility=false}

\usepackage{graphics,graphicx,epsfig,ulem}	% Makes sure all graphics works
\usepackage{amsmath} 						% Adds mathematical features for equations

\usepackage{fancyhdr}

\def\thesection{\arabic{section}}
\def\thesubsection{\alph{subsection}}

\def\bibsection{\section*{References}}        % Position reference section correctly

%%%%% Document %%%%%
\begin{document}                     

\title{Testing the Milankovitch-Croll hypothesis using $\delta^{18}$O foram data} 
\maketitle
%\thispagestyle{plain} % produces page number for front page

\vspace{-4ex}

The Milankovitch-Croll hypothesis suggests that changes in the Earth's orbit around the Sun leads to changes in Earth's planetary climate. If the orbit varies then there will be fluctuations in solar insolation which in turn has an effect on the Earth's climate \cite{ruddiman_climate}. We can access and view this climate change through analysing proxy data. Through the usage of deep ocean sediment cores and Earth orbit data, we can explore this hypothesis and attempt to discover if it has any validity or scientific basis. \\

The importance of sediment cores arises from the fact that it contains a reliable trace for measuring the $\delta^{18}$O content during a specific period of Earth history. 

\newpage

%\twocolumngrid

\bibliographystyle{unsrt}
\bibliography{practical_2}

\section*{Figures and tables}

\end{document}